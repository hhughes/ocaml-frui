\chapter{Implementation}
A series of problems I will solve to determine what type of problem ocamljs and froc are most useful for.
\section{Log Viewer}
Problem: \emph{Reading log files is difficult and confusing when we have multiple threads. Design a system using ocamljs and froc which could replace the log module in a program to display the messages in a more helpful way.}

One way to display this information is to show a time-line for each thread. Debug messages can be shown as points on each thread line and function enters and exits can also be added to show which function we are in at any point.

\subsection{JSON}
ocamljs creates web applications, therefore the log viewer will also be one of these. JSON (JavaScript Object Notation) shall be used to communicate with the log viewer interface. JSON is used because when a string of JSON is passed to the JavaScript \emph{eval} function it is converted to the correct JavaScript object.

This is fairly straight forward in JavaScript but doing the same in OCaml is slightly more tricky. In JavaScript objects are collections of key-value pairs (usually implemented as a hash table to provide a fast lookup) where the keys are strings and the values can be any object. New key-value pairs can be added at runtime by setting unused field values, if an unused key is referenced then the null object is returned. This is very \emph{unOCaml}-like behaviour. In OCaml the field would have to be explicitly declared in the object definition. However, because we know the JSON schema for our objects we can make a class for it.

There are two ways to convert the JSON string into a JavaScript object. The first is to parse the string manually and set the values of mutable fields on the object as we go along. The problem with this is that unmarshalling the JSON would have some tricky corner cases, such as string padding, what to do with (and how to detect) invalid JSON? A better solution is to use the JavaScript \emph{eval} to convert the JSON to an object. This issue with this is how do we access the fields on the object?

With OCaml not being as popular as a language like C there are many libraries which would be useful for an OCaml program which has been implemented in C. Sometimes it isn't worth porting the C code to OCaml so an interoperability layer has been added to OCaml which provides a way to interface with C libraries. Communication between OCaml and C is achieved using \emph{external} function declarations. These are a very similar to prototype functions found in C header files. External functions have the following syntax:

\begin{center}
\texttt{external caml\_name : type = "C\_name"}
\end{center}

There are special prefixes for the function name string which are handled by the ocamljs compiler to provide JavaScript field accessors and method calls. Figure \ref{external} shows a table listing these.

With this we can create a function which will apply a string to the \emph{eval} function and return us the JavaScript object wrapped by a dummy OCaml class. For example in this case the class has type \emph{msg} so the type of the external function for converting a JSON string to a msg is \texttt{string -> msg}. The code for this method is as follows:

\begin{center}
\texttt{external parse\_json : string -> msg = "@eval"}
\end{center}

In order to access fields we use the \emph{.} prefix for the external function and pass in the JavaScript object. OCaml has no way of checking if these types are going to correct at runtime so an exception will be raised if they are wrong. A pitfall discovered using this is that JSON uses strings to represent floating point numbers, if the external function is typed \texttt{msg -> float} this will compile but using the value will cause undetermined errors at runtime. OCaml will treat the value in memory as a float when it is really a string.

It is unlikely there will be just one message ready each time we poll the server. Therefore it would be sensible to store a queue of messages at the server and flush the whole queue at once. The JSON would then be an array of objects. ocamljs provides a \emph{JavaScript} module which provides function wrappers around some of the common functions and objects built into JavaScript. One such is \emph{js\_array}, an object which can represent a JavaScript array object. If we are parsing a JSON array of objects we can use the type \texttt{string -> msg js\_array} to return and object which represents an array.

We can convert a \emph{js\_array} into a list in \texttt{$O(n)$} with the following function.

\begin{alltt}
let rec js\_array\_to\_list xs =
  if xs\#\_get\_length > 0 then
    xs\#pop :: (js\_array\_to\_list xs)
  else []
\end{alltt}

\begin{figure}
  \centering
  \begin{tabular}{|l|l|}
    \hline
    \textbf{prefix} & \textbf{compiles to}\\ \hline
    \texttt{\#} & method call\\ \hline
    \texttt{.} & read property\\ \hline
    \texttt{=} & assign property\\ \hline
    \texttt{@} & call built-in function\\ \hline
  \end{tabular}
  \caption{ocamljs external function prefixes}
  \label{external}
\end{figure}

\subsection{froc}
\emph{froc} is used in the log viewer to make the div elements (which show messages, functions and threads) re-align themselves when the time range changes.

In the first implementation the time range of the visualiser only changes when new messages arrive. The start time is set to the earliest known message and the end time is when the latest message was generated. The ends of the time range represented by two \emph{froc} variables. When a new message arrives, if it is newer than the newest message or older than the oldest then update these variables. A message is then turned into a div element. A function which lays the div in the correct place on the page is bound to the two \emph{froc} variables which represent the time range. Now whenever the variables are updated each of the divs bound to these have their callbacks run and the divs are realigned.

\subsection{Controlling the Time Range}
One of the requirements was to provide a way to zoom into particular ranges. The simplest way to provide this functionality is to use two input boxes, one which represents the time value for the leftmost message (minimum) and the other for the rightmost value (maximum). When the values in the input boxes change the two variables which represent the time range are updated. Alternatively if a new message arrives and one of the time range variables update then the input boxes will now not correspond to the correct value of the maximum of the time range. Therefore this has to update the values in the input boxes too. Figure \ref{spinners} shows the dependency graph for these controls.

\begin{figure}
  \centering
  \includegraphics[scale=0.5]{graphs/spinners.png}
  \caption{Dependency Graph for Spinners}
  \label{spinners}
\end{figure}

There is a cycle in the dependency graph. This usually would mean that once an update gets into the cycle it would never terminate. However, because froc only sends an update when the value actually changes the cycle gets broken. This means no defensive programming is required to prevent cycles in the dependency graph, this is all managed by \emph{froc}.

\section{cohttp-server}
To retrieve JSON data the log viewer interface needs to use the \emph{HTTP GET} command. This simply requests a page from the server. This page need not be a real file and we can modify the HTTP server to generate some JSON on the fly to give back.

The HTTP server implementation I used is called cohttp-server. This is because it is written in OCaml and the source code is free to use.

\subsection{State Machine}
In order to generate some interesting and believable data to test the log viewer on I am going to use a state machine to determine the next state of each thread. Each thread must be one of the following states (visualised in Figure \ref{state})
\begin{itemize}{}
\item Started
\item Running
\item FunEnter
\item FunExit
\item Msg
\item Finished
\item Stop
\end{itemize}

\begin{figure}
  \centering
  \includegraphics[width=10cm]{graphs/state.png}
  \caption{State machine diagram for test data}
  \label{state}
\end{figure}

\subsection{JSON generation}
JSON is generated using the \emph{json-wheel} OCaml module

\section{froc Extension: froc-lists}
\subsection{Motivation}
Lots of events

If we have too many to store we can poll for more

Need to store these events in a list object

If we used \texttt{\textnormal{$\alpha$} list behavior} then we would have to rebuild all the objects from scratch when a new item was added to the list.

When we have a list we want to be able to apply a lift or bind to all the current elements of the list and all future ones. To simplify lets assume we are only ever going to need to push one element to the list of pop one element off at a time. We will also need a method to give us the list in the form of an OCaml list. Finally we will want to add to both the end and the start of the list if we are polling for data.

\subsection{Implementation}

The underlying data structure is two lists, one is the first half of the list (\texttt{first}), the other is the second half(\texttt{last}). Items are arranged in these lists such that pushing to the list \texttt{first} will add items to the beginning of the list and pushing to \texttt{last} will add items to the end of the list.

To return the object as an OCaml list the method \texttt{list} is implemented as follows:

\texttt{method list = List.rev\_append (List.rev first) (List.rev last)}

This runs in \texttt{$O(n)$}.

\subsection{Extensions}
Unlike froc behaviours we cannot bind a function to when the list itself changes.

\section{Pie Chart}
Problem: \emph{When we have data being processed in real time it would be nice to have a widget which showed some property of that data and which updated as new data comes in}

Map strings (which could represent message types for example) to integers which get incremented each time an occurrence of this happens. Use froc to redraw the pie each time the counters change.

\subsection{HTML Canvas Element}
The Canvas element is a new element type added to HTML5. It provides a bitmap image and functions to draw 2D shapes and lines. It is much more flexible than drawing using HTML divs however removing objects from the scene requires erasing and redrawing whole sections.

\subsection{Rendering the Pie}
The pie chart shows the proportions of each type of message so will have to update each time a new message arrives. The canvas element provides an arc drawing function. The start of the arc will be wherever the last one finished (or 0 if it is the first piece) and the end of the arc will the proportion of total items that piece represents around the circle ($2\pi$ radians). Unfortunately when any part of the pie chart has to update the whole canvas has to be erased and the pie redrawn from scratch so the self adjusting properties of froc cannot be exploited.

\subsection{Word Cloud}
Problem: \emph{Similar to the pie chart, display the most popular words}

The word cloud is very similar to the pie chart. Instead of counting object in predefined categories (such as message type) it count occurrences of words. There is not limit to the number of objects we are counting. The data structure to use in this case is a hash table. Use the word that is being counted as the key and the number of past occurrences as the value. When a new word appears the lookup in the hash table fails and a new entry can be added. This algorithm runs in \texttt{$O(1)$} however the space requirements are \texttt{$O(n)$} and the hash table object will require \texttt{$O(n)$} time each time it grows.

To represent the data the words are drawn on a canvas with the font size (in pixels) set to the relative proportion that word is used.

\section{Dataset Graph}
Problem: \emph{We would like to compare some data which varies in multiple dimensions on a 2D graph.}

For example the data used in \emph{The Joy of Stats}\footnote{\url{http://www.gapminder.org}}. Data varies over time which is why froc is suited for this application. Each data point is represented by a HTML div element which realigns itself when the \emph{current} data changes (as we run through time and pick different data points).

\subsection{Data Sources}
There are lots of data set available on the internet for example Twitter provide numerous data sets providing information about recent Tweets, posting users etc. There are several problems with this. The number of items returned each time is limited, the number of requests in a given time period are limited and each record is very large so contain a lot of unneeded data. Live data is quite often not very useful. With the Twitter data some of the time fields are missing etc.

Historical data is much better because it can use usefully formatted and collated. The World Bank provides an API for a data set covering economics and development statistics for each country. It provides a RESTful interface and the data can be presented in JSON format. This data can be used to recreate a graph similar to that used in The Joy of Stats, which shows GDP against life expectancy all over time. To prevent having to request the same data over and over again from the World Bank web site, a snapshot can be made by using the \emph{wget} Unix tool. A script can be written to get all the pages of data required.

\subsection{Data Structures for the Axes}
After parsing all the data there needs to be an effective way to inject the current data value for each axis into each data point. It seems sensible to use two froc variables for each data-point, one for each axis. As the time changes these variables get updated with the correct value from the dataset. The data structures for storing both the data and the froc variables need to be chosen carefully.

When the time value changes the whole set of data for that year is needed. Therefore it makes sense to use a hash table to store the data which is index by year. The value at each index is a list of key-value pairs with the key being the country ID and the value being the data value for that country in that year. Next the correct froc variables need to be found and updated with the new values. As the program iterates through the list of values the ID for the country is known at each point. Therefore if the froc variables were also stored in a hash table but this time indexed by country ID the lookup time to update the values will be constant. Therefore the total time to change the year should be \texttt{$O(n)$} where \emph{n} is the number of countries. 

\section{Heat Map}
Problem: \emph{How do we display data about energy usage over time for a number of rooms in a building?}

A heat map can be used to show the relative amounts of energy usage in different rooms at a given time. The higher the value for that room the \emph{hotter} the colour of that part of the map. A heat map is a graph with the x and y axes have been changed to represent physical dimensions. The values for a and y will not change so it's implementation is simpler than for the dataset graph.

\subsection{Rendering the Map}
From using the canvas element to draw the pie chart and word cloud we saw that drawing using canvases provided lots of flexibility but updating only the parts which have changed is difficult. If div elements were used, however, then the browser handles redrawing them itself. The problem with using div elements is that they are rectangular and rooms are often not. This could be solved by using multiple div elements which when coloured the same will form the shape of the room. Because this is aesthetics this project will just use a square in the middle of the room to represent the colour for the whole room.

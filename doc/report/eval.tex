\chapter{Evaluation}

\section{ocamljs vs JavaScript}
\emph{At what point does writing ocamljs become more effective than js?} Probably once there are multiple developers working on the same piece of web application code. JavaScript quickly becomes unmanageable but OCaml uses modules and classes which can all be independent of each other. This makes it a lot more useful for large scale projects. Also because of the strong type checking, programming errors can easily be identified. This is not so easy with large pieces of JavaScript code where they can become lost and take a long time to debug.

However if the application is fairly small (~100 lines of JavaScript) it would almost certainly be quicker and easier to write it straight in JavaScript, ocamljs is very clunky and repetitive for doing things one off such as setting styles and adding text to elements.

Sometimes the JavaScript interpretation of the code does not do what is expected. The OCaml compiles and the bytecode version would do the correct thing but the JavaScript doesn't work. This is often due to limitations in the compilation of OCaml into JavaScript. One example occurred when using the Hashtable module. At first the data structure for the graph control used a hash table indexed by year (an integer). The OCaml hash table has the following type:

\begin{center}
\texttt{type ($\alpha$, $\beta$) t}
\end{center}

This means that both the key and the value can be of any type (and can be different types from each other). However when the OCaml is compiled into JavaScript it only works if the keys are strings. In JavaScript hash tables are implemented as standard JavaScript objects. These objects can have arbitrary fields so behave exactly like hash tables indexed by strings. This does work with integers because the browser interprets them as their string value. However, the ocamljs compiler

\section{Where froc was useful}
\subsection{Log viewer}
Sort of...
\subsection{Pie chart}
No. We had to do just as much work anyway.
\subsection{Word Cloud}
No. We had to do just as much work anyway.
\subsection{Dataset Graph}
Yes. It was very effective at moving the data points around when needed. It does the minimum work necessary.
\subsection{Heat map}
Yes. It only changes the colours when needed. Again doing the minimum work.

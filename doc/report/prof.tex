\chapter{Proforma}
\begin{center}
\begin{tabular}{r l}
Name and College: & Henry Hughes, Jesus College\\
Project Title: & Functionally Reactive Web Applications\\
Examination and Year: & Computer Science Tripos 2011\\
Word Count: & Approx. 9,400 words\\
Project Originator: & Dr A. Madhavapeddy\\
Project Supervisor: & Dr A. Madhavapeddy\\
\end{tabular}
\end{center}

\section*{Original Aims}
The purpose of this project is investigate the usefulness of Functional Reactive Programming (FRP) when designing and implementing web applications. It will focus on just one method of FRP, using a modified version of the OCaml compiler called \emph{ocamljs} and the library \emph{froc} which is used for reactive programming in OCaml. The project will look how at the speed of applications produced using FRP compare to JavaScript implementations. It will also look at how the language properties of OCaml help the developer develop code which will be type safe and produce the correct output. % 94 words

\section*{Work Completed}
The project implemented three applications using \emph{ocamljs} and \emph{froc}. The first was a tool for rendering log files for threaded programs in a visual format. \emph{froc} was used to redraw the user interface elements whenever the state of the application changed. The second rendered a graph of data with two variables which varied over time. As the time viewed changed, \emph{froc} would update the values for the data points and then reposition them. The final application was a heat map of energy usage in a building. \emph{froc} sets the appropriate colour for the rooms as the time value changed. % 95 words


\section*{Project Difficulties}
None.

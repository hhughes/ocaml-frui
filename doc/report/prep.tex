\chapter{Preparation}

\section{OCaml}
Objective Caml (\emph{OCaml}) is a variant of the Caml language. It is an extension of Caml which adds an object-oriented layer and module system. It is designed for use in developing commercial systems and is the most popular Caml derivative.\footnote{\url{http://caml.inria.fr/ocaml/index.en.html}}

Caml is a general purpose language which supports functional, imperative and object-oriented programming styles. It features a powerful type system which uses parametric polymorphism and type inference. This allows methods to be designed without having to explicitly declare the types of parameters or the result so functions can be reused on many different types of inputs. It also has pattern matching which can be used to direct control flow of the program through functions depending on their inputs.

\section{JavaScript}
JavaScript is mainly used for client-side scripting on web pages. It is run on the local machine inside the browser environment and can access and modify the Document Object Model (\emph{DOM}) which is the browsers representation of the current web page. For this reason it can be used to make interacted web user interfaces and dynamic pages. In some ways JavaScript is very similar to OCaml. In both languages functions are first class objects, they can be assigned to variables, passed as function parameters and invoked. However the typing systems of OCaml and JavaScript are very different. OCaml types start off generic and get more specific each time they are used. In JavaScript variables themselves do not have a type and so can be reassigned to a value of any type. This can lead to programming errors because you are not guaranteed to know the type of your object at any point. Another difference is that OCaml is checked for errors at compile-time where as JavaScript is checked at run-time. If a piece of code is not executed in a test-run you cannot be sure that it will succeed.

\section{Web Applications}
JavaScript is the only client-side scripting language supported by the most popular web browsers so web applications have to use it. Developing in JavaScript, as with all scripting languages, is fairly rapid but as the code base grows keeping track of types of variables and what code gets executed when soon becomes unmanageable. It would be nice if JavaScript could provide the same guarantees for the web application as for programs compiled using OCaml.

\section{ocamljs}
ocamljs is a modified version of the OCaml compiler. It uses the standard OCaml compiler up to the point when lambda code is generated from the abstract syntax tree. Figure \ref{lambda} shows a simple OCaml program and it's corresponding lambda code. After this it transforms the lambda code into JavaScript. Functions and exceptions map simply into JavaScript. Integers and floats can be represented as a JavaScript \emph{number} and boolean by the JavaScript \emph{bool} or a JavaScript \emph{number}. The standard library functions have been reimplemented in a static JavaScript file.

\begin{figure}
  \begin{tabular}{| p{3.5cm} | p{7.8cm} |}
    \hline
    \textbf{OCaml (simple.ml)} & \textbf{Lambda}\\ \hline
    \begin{alltt}
let f a b = a+b
let three = f 1 2;;
    \end{alltt}
    &
    \begin{alltt}
(setglobal Simple!
  (let (
    f/58 (function a/59 b/60 (+ a/59 b/60))
    three/61 (apply f/58 1 2))
    (makeblock 0 f/58 three/61)))
    \end{alltt} \\ \hline
  \end{tabular}
  \caption{OCaml example code and it's lambda representation}
  \label{lambda}
\end{figure}

\subsection{Function Applications}
Function applications are a bit more tricky. Functions in JavaScript require the correct number of arguments but with OCaml functions can receive more (partial application) or less arguments (tail calls). When we have a partial application we want to return a closure and when we have a tail call we want to apply the extra arguments to the result. This is solved using Simon Peyton Jones' \emph{eval-apply} method.

\subsection{Eval-Apply}
With this scheme the caller is responsible for providing the correct number of arguments to a function. If there are not enough a closure has to be created and if there are too many the left over arguments are applied to the result of the function. This is implemented using the \emph{apply} function outlined in Figure \ref{eval-apply}.

\begin{figure}
  \begin{alltt}
f a\subs{1} ... a\subs{n} -> apply\subs{n}(f, a\subs{1}, ..., a\subs{n})

apply\subs{n} = \lam f x\subs{1} ... x\subs{n}
  match arity(f) with
    | 1   -> apply\subs{n-1} (f(x\subs{1}), x\subs{2}, ..., x\subs{n})
    | ...
    | n-1 -> apply\subs{1} (f(x\subs{1}, ..., x\subs{n}), x\subs{n})
    | n   -> f(x\subs{1}, ..., x\subs{n})
    | n+1 -> papp\subs{n+1,n}(f, x\subs{1}, ..., x\subs{n})
    | n+2 -> papp\subs{n+2,n}(f, x\subs{1}, ..., x\subs{n})
    | ...

papp\subs{p,q} = \lam f x\subs{1} ... x\subs{q}. (\lam x\subs{q+1} ... x\subs{p}. f(x\subs{1}, ..., x\subs{p}))
  \end{alltt}
  \caption{Eval-apply implementation}
  \label{eval-apply}
\end{figure}

\subsection{ocamljs Example}
For a small example consider a simple function which takes two arguments and returns the summation of them and then an application of this function. Figure \ref{example} shows the OCaml code and compiled JavaScript for the example.

\begin{figure}
  \begin{tabular}{| p{4cm} | p{7.3cm} |}
    \hline
    \textbf{OCaml} & \textbf{JavaScript}\\ \hline
    \begin{alltt}
let f a b = a+b
let three = f 1 2;;
    \end{alltt}
    &
    \begin{alltt}
function () \{
  var f\$58 =
    _f(2, function (a\$59, b\$60) \{
      return a\$59 + b\$60;
    \});
  var three\$61 = _(f\$58, [ 1, 2 ]);
  return \$(f\$58, three\$61);
\}
    \end{alltt} \\ \hline
  \end{tabular}
  \caption{OCaml source and JavaScript output example}
  \label{example}
\end{figure}

\section{froc}

\emph{froc} is and OCaml library for functional reactive programming in OCaml. It uses self adjusting computation to push updates to input variables through data paths in the program. Froc implements a syntax extension which is used to build dependency graphs in the program. Dependencies can be chained so that expressions use the result of other expressions. The self adjusting computation ensures that when an input changes, the least about of computation is done: expressions who's inputs don't change are not recomputed.

\subsection{Syntax}
Froc introduces a wrapper around values called a \emph{behavior}. One of the methods froc adds is called \emph{bind}. There is also an additional syntax expression, \emph{>>=} which is short for bind. The bind function links an input behavior to a callback function and returns another behavior (the output). The callback function is run with the new value of the input behavior each time it is updated. The value the callback returns is the new value for the output behavior. Figure \ref{add_code} shows an example of froc and Figure \ref{add_graph} shows the dependency graph that froc holds internally.

\subsection{Dependency Graphs}
Figure \ref{if_graph} shows a more complex dependency graph. This example illustrates two properties of froc. The first is lazy evaluation. If we can calculate the value of \emph{z} and the result of \emph{z=0} first then we may not need to calculate \emph{x+y}. Preventing calculations for unused results reduces the running time of a program. The second feature is the essentially the same but is used to avoid evaluating expressions which could cause exceptions which are expensive to handle. Correct control flow in a program ensures that the lazy evaluation will only execute expressions which will succeed.

\begin{figure}
  \begin{alltt}
let x = return 1
let y = return 2
let z = return 3

let i0 =
    x >>= fun x ->
        y >>= fun y ->
            return (x + y)
let ans =
    i0 >>= fun i0 ->
        z >>= fun z ->
            return (i0 + z)
  \end{alltt}
  \caption{Addition using froc}
  \label{add_code}
\end{figure}

\begin{figure}
  \centering
  \includegraphics[scale=0.5]{graphs/addition.png}
  \caption{Example dependency graph}
  \label{add_graph}
\end{figure}

\begin{figure}
  \centering
  \includegraphics[scale=0.5]{graphs/if.png}
  \caption{Example if-statement dependency graph}
  \label{if_graph}
\end{figure}

\section{Version Control}

Version control is very important for a software project. It involves breaking the project into a number of changesets, each of which is given a description, and the idea is that the code is in a consistent state before and after each commit. This is often used in conjunction with pushing changesets to a remote server which is regularly backed up. As a consequence if files get corrupted, deleted or changed in such a way that work has been undone they can be reverted to a working copy.

There are many version control systems. Git is the one used for the ocamljs and froc projects. It has the functionality required above and there is a free to use service run by \emph{GitHub}\footnote{\url{http://www.github.com}} on the condition at your code is publicly viewable and anyone can fork your repository. The repository for this project can be found at \url{https://github.com/hhughes/ocaml-frui}.

\section{Compiling the Project}

Use the \emph{GNU Make}\footnote{\url{http://www.gnu.org/software/make/}} system. Make is essentially a wrapper around a bash script. It provides a simple way for anyone to compile source code because it is widely used. It also uses dependencies to only compile those parts of the project which have changed, reducing the compile time of the project during development.

\section{HTTP Server}

An HTTP server is required to serve up a web page. At first a stock web server, such as \emph{Apache}, seemed like a good idea because it requires minimal setup. This is good for serving static content (such as HTML pages and JavaScript files) but dynamic content (such as time-dependent JSON messages) proves more tricky. In order to deliver interesting JSON data some server code is required. The two options are to use some sort of server side scripting which Apache can execute, although this requires learning a new language such as \emph{PHP} or to find an implementation of an HTTP server in a language this project is already using (such as OCaml) and modify it such that JSON data can be generated at run-time and delivered to the client.

Using an OCaml HTTP server is the more sensible solution because it gives the greatest amount of time and lets me concentrate on writing the JSON generating code rather than getting stuck learning a new syntax. The OCaml web server I shall use is \emph{ocaml-cohttpserver}\footnote{\url{https://github.com/avsm/ocaml-cohttpserver}}.

\section{Dependencies}
ocamljs \& froc:
\begin{itemize}
\item ocaml source
\item findlib
\item ulex
\item camlp4-extra
\end{itemize}
cohttp-server:
\begin{itemize}
\item lwt
\item react
\item libev
\item cohttp
\end{itemize}
ocaml-frui:
\begin{itemize}
\item ocamljs
\item froc
\item cohttp-server
\item json-wheel
\end{itemize}

\section{Requirements Analysis}
